\section{Механика}
Имеет основное разделение на \textbf{Кинематику} и \textbf{Динамику}.

\subsection{Кинематика}
раздел механики, изучающий движение материальных объектов без учёта их масс и действующих на них сил.
\par\medskip

\subsubsection{Термины и определения}

\md{Механическое движение} -- изменение положения тела относительно других тел с течением времени.
\par\medskip

\md{Материальная точка} -- это модель тела, размерами которого в данных условиях можно пренебречь.
\par\medskip

\md{Система отсчёта} -- тело отсчёта, система координат и часы, связанные с ним.
\par\medskip

\md{Траектория} -- линия, по которой движется тело.
\par\medskip

\md{Путь ($S$)} -- величина, характеризующая длину траектории.
\par\medskip

\md{Перемещение ($\vec{S}$)} -- вектор, соединяющий начальное и конечное положение тела.
\par\medskip

\md{Скорость ($\vec{\upsilon}$)} -- векторная величина, показывающая быстроту изменения координаты.
$$|\vec{\upsilon}| = \frac{S}{t}$$

\md{Мгновенная скорость} -- скорость в данный момент времени.
\par\medskip

\md{Средняя скорость} -- отношение всего пути ко всему времени.
\par\medskip

\md{Ускорение ($\vec{a}$)} -- быстрота изменения скорости.
$$|\vec{a}| = \frac{\Delta\upsilon}{t}$$

\subsubsection{Виды движения}
\begin{enumerate}
	\item Равномерное прямолинейное ($a = 0$):
	$$x(t) = x_0 + \upsilon t$$

	\item Равноускоренное движение ($a\neq 0$):
	$$x(t) = x_0 + \upsilon_0 t + \frac{a t^2}{2}$$
	$$\upsilon(t) = \upsilon_0 + at$$
	$$\frac{\upsilon^2(t)-\upsilon^2_0}{2a} = x(t) - x_0$$

	\item Свободное падение ($a = g$):
	\item Движение по параболе($a_y = g, ~\upsilon_x = const$):
	\item Движение по окружности:
	\par\medskip

	\md{Угловая скорость ($\omega$)} -- скорость изменения угла.
	$$\upsilon = \omega R$$
	\md{Центростремительное ускорение ($a_ц$)} -- ускорение сохраняющее траекторию движения по окружности.
	$$a_ц = \omega^2 R = \frac{\upsilon^2}{R}$$
\end{enumerate}

\subsubsection{Сложение скоростей}

\ml{
	Закон Галилея (нерелятивиское сложение скоростей)
}{
	Скорость тела в абсолютной системе отсчёта складывается из скорости тела в подвижной системе и из скорости самой системы.
}

$$\vec{\upsilon_а} = \vec{\upsilon_о} + \vec{\upsilon_п}$$


\subsection{Динамика}
